%1. La memoria incluirá una portada portada normalizada con la siguiente información: título en castellano, título en inglés, autores, profesor director, codirector si es el caso, curso académico e identificación de la asignatura (Trabajo de fin de grado del Grado en - nombre del grado correspondiente-, Facultad de Informática, Universidad Complutense de Madrid). Los datos referentes al título y director (y codirector en su caso) deben corresponder a los publicados en web de TFG, según lo indicado en el punto 6 de la sección III de esta normativa. 
% 2. en cada apartado
% 3. La memoria constará de un mínimo de 25 páginas para los proyectos realizados por unúnico estudiant

\documentclass[12pt , a4paper]{article}
\usepackage[spanish, english]{babel} 	% language
\usepackage{tocloft}					% table of contents

% for sections
\renewcommand{\cftsecleader}{\cftdotfill{\cftdotsep}} 

% images
\usepackage{graphicx}
\graphicspath{ {./images/} }

% text columns
\usepackage{multicol}
\setlength{\columnsep}{1cm}



\title{Instrumento virtual basado en samples para estaciones de trabajo de audio digital\\Sampler plugin for digital audio workstations}
\author{Juan Chozas Sumbera}


\begin{document}
	\maketitle
	\begin{center}
		Coordinado por\\Jaime Sánchez Hernández y Miguel Gómez-Zamalloa Gil
		\newline
				
		\includegraphics[width=\textwidth]{logo_UCM.jpg}
	\end{center}

	
	
	% 2. La memoria debe incluir la descripción detallada de la propuesta hardware/software realizada y ha de contener: 
	% a. un índice, 
	\newpage
	\tableofcontents
	
	% b. un resumen y una lista de no más de 10 palabras clave para su búsqueda bibliográfica, ambos en castellano e inglés, 
	\newpage
	\section{Abstract}
	* separar en dos pg (una para cada idioma) y describir: MIDI, DAW, PLUGINS, SYNTH \\ seguir con este abstract

	\begin{multicols}{2}
	A sampler is a synthesizer that uses samples rather than oscillators to produce sound.  The sources of the sounds it produces are samples, which can be described as clips of audio that are stored in a digital format within the sampler's memory. You're likely to find that hardware and software samplers offer sets of "stock" samples that are loaded by default. The option to load samples from an external source such as an SD card is also common, just like recording external input from a recording device such as a microphone. As is the case with non sample based synthesizers, there are several ways that the samples can be manipulated. Once samples are loaded, the sampler can make the notes on a keyboard trigger the samples so users can play them to compose music. This sampler, in particular, takes the form of a plugin for digital audio workstations. This means that the the digital audio workstation routes triggering events into the plugin, which in turn processes the input to produce a stream of audio that is returned to the host as sound.\\


	Un sampler es un sintetizador que produce sonido mediante samples en lugar de generar sonidos con osciladores. Las fuentes del sonido que produce son samples: muestras de sonido grabado en forma digital que se almacena en en la memoria del sampler. Muchos samplers vienen con una serie de samples cargados por defecto, pero siempre ofrecen la opción de cargar samples distintos. Esto se puede hacer a través de una unidad de almacenamiento, como una tarjeta SD, o con una señal externa usando un dispositivo con capacidad de grabación como la que tiene un micrófono. Igual que con los sintetizadores, hay múltiples formas de manipular los samples para producir sonidos personalizados. Una vez cargados los samples, se pueden asociar a las notas de un teclado para accionar su reproducción y permitir que el usuario componga música. Este sampler está implementado como plugin para estaciones de audio digital, y funciona tomando como entrada disparadores de la estación de audio digital para devolver las notas recibidas en forma de sonido. 
	\end{multicols}
	\subsection{Key words}
	sound, sampler, sample, sampling, digital instrument, MIDI, DAW, plugin, VST, VST3, synthesizer\\


	\section{Resumen}
	* separar abst en dos pg (una para cada idioma)
	
	\subsection{Palabras clave}
	sonido, muestra, muestreo, instrumento digital, sintetizador
	


	% c. una introducción con los antecedentes, objetivos y plan de trabajo, 
	\newpage
	\section{Introduction}
	\subsection{History of sampling}
	
	The technique of sampling dates back to the 1960s when recordings were captured on tape. Thanks to a hardware design that made contact between the playback head and different sections of a moving tape, musicians could play the distinct recordings on keyboards of instruments such as the Mellotron. During the 1980s, the popularity of drum machines increased significantly. Many drum machines were sample-based, i.e. they created sounds using digitally stored samples. The alternative was to synthesize sound in an analog fashion. In 1988, the first Music Production Center (MPC) by Akai was made available to the public: a sample-based music workstation that was capable of arranging samples of all lengths in its sequencer to produce full fledged music tracks without the need for additional instruments or hardware. In a recipe for music, an MPC could be the only ingredient, reaching a vast number of musicians because of its affordability in comparison to previous means of music production. It is a tool that gave artists a new way to create music, a technique that has been a foundation to several genres and highly influential to music as a whole.
	\par	
	In a world of highly complex information systems, digital music production has been developed thoroughly. One can have an extensive range of digital synthesizers, samplers, and effect chains all working simultaneously in computer software. A Digital Audio Workstation (DAW) provides a playground that routes audio between racks of components and effects. With support for plugins, such as instruments or effects, the software's capabilities can be extended to broaden a musician's repertoire for music production. 
	
	\subsection{Objectives}
	Most DAWs have built in samplers, or tools that allow a user to sample sound and directly manipulate it so as to employ sampling techniques. In most cases, these tools are more than capable of providing a means to translate these techniques into action, however, depending on the DAW, you may be hindered due to the program's presentation. Whether it be the user interface, the imposed workflow, or the minimum amount of steps required to reach your sampling goals, it is likely that the process involves inconveniences. 
	
	* workflow es clave, hay que poder hacer lo q uno quiere. si se quita lo innecesario, quedando lo esencial, se pierde menos tiempo a la hora de crear en lo q es un proceso creativo
	*explain chop and assign to keyboard as prerequisite to playing around
	
	The aim here is to reduce the amount of user interaction required to translate an idea into sound. For the simple technique of sampling, the means by which you achieve a minimum setup with which you can play around and manifest ideas should be straightforward. In addition, the user should also be able to make fast corrections and tweaks to further develop the instrument. Once again, it is useful if this does not imply having to restart the whole process in a separate instance of the instrument, for example.
	
	% d. resultados y discusión crítica y razonada de los mismos, con sus conclusiones,
	\newpage
	\section{Results}
	\subsection{Conclusion}
	 
	\newpage
	% e. bibliografía.
	\section{Bibliography}
	
\end{document}