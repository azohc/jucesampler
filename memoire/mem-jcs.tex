%1. La memoria incluirá una portada portada normalizada con la siguiente información: título en castellano, título en inglés, autores, profesor director, codirector si es el caso, curso académico e identificación de la asignatura (Trabajo de fin de grado del Grado en - nombre del grado correspondiente-, Facultad de Informática, Universidad Complutense de Madrid). Los datos referentes al título y director (y codirector en su caso) deben corresponder a los publicados en web de TFG, según lo indicado en el punto 6 de la sección III de esta normativa. 
% 2. en cada apartado
% 3. La memoria constará de un mínimo de 25 páginas para los proyectos realizados por unúnico estudiant

\documentclass[12pt , a4paper]{article}
\usepackage[spanish, english]{babel} 	% language
\usepackage{tocloft}					% table of contents

% for sections
\renewcommand{\cftsecleader}{\cftdotfill{\cftdotsep}} 

% images
\usepackage{graphicx}
\graphicspath{ {./images/} }

% text columns
\usepackage{multicol}
\setlength{\columnsep}{1cm}



\title{Instrumento virtual basado en samples para estaciones de trabajo de audio digital\\A sampler plugin for digital audio workstations}
\author{Juan Chozas Sumbera}


\begin{document}
	\maketitle
	\begin{center}
		Jaime Sánchez Hernández\\
		Miguel Gómez-Zamalloa Gil
				
		\includegraphics[width=\textwidth]{logo_UCM.jpg}
	\end{center}

	
	
	% 2. La memoria debe incluir la descripción detallada de la propuesta hardware/software realizada y ha de contener: 
	% a. un índice, 
	\newpage
	\tableofcontents
	
	% b. un resumen y una lista de no más de 10 palabras clave para su búsqueda bibliográfica, ambos en castellano e inglés, 
	\newpage
	\section{Description}

	\begin{multicols}{2}[\subsection{Summary}]
	Essentially, the plugin is a digital instrument that makes sounds using samples. Starting with an arbitrary audio sample, one can create sections within the sample. The, the sections can then be played back with a MIDI controller, or arranged in a pattern in the host's sequencer. 

	El plugin es un instrumento digital que utiliza samples para crear sonido. Partiendo de un sample, podemos asignar secciones del sample a notas MIDI. El plugin puede reproducir estas subsecciones, y también se pueden organizar en el secuenciador del host que carga el plugin.
	\end{multicols}

	\subsection{Key words}
		sound, audio, sample, sampler, plugin, vst, juce, daw, 

	
	% c. una introducción con los antecedentes, objetivos y plan de trabajo, 
	\newpage
	\section{Introduction}
	
	% d. resultados y discusión crítica y razonada de los mismos, con sus conclusiones,
	\newpage
	\section{Results}
	\subsection{Conclusion}
	 
	\newpage
	% e. bibliografía.
	\section{Bibliography}
	
\end{document}